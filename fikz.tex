\documentclass{article}
\usepackage[margin=1.2in]{geometry}
\usepackage{tikz, fikz, hyperref}

\title{\vspace{-7mm}The \textsf{fikz} package\footnote{This file has version number v1, last revised on 2021/08/31.}}
\author{Benjamin Bernard}

\begin{document}
\maketitle

\vspace{-1mm}
\begin{abstract}
This \LaTeX\ package changes the basic overlay behavior of \verb|tikz| macros from the \verb|\only| functionality to the \verb|\onslide| functionality, thereby solving the ``jumping image'' issue.
\end{abstract}

\section{Introduction}

The \verb|beamer| document class introduces overlay functionality to gradually reveal information on frames. The standard overlay functionality added to the \verb|tikz| macros is the \verb|\only| functionality, which reserves space for the generated path only on the slides, on which the paths are visible. This causes gradually revealed \verb|tikz| pictures to ``jump around'' from slide to slide if the picture's bounding box changes. This package changes the default overlay functionality of the \verb|tikz| macros to the \verb|\onslide| functionality, which reserves space for the generated path on all frames of the slide. Because I did not find a neat solution on \href{https://tex.stackexchange.com}{\texttt{tex.stackexchange.com}}, I posted this problem at \url{https://tex.stackexchange.com/questions/599624/replace-tikzs-standard-overlay-specification-from-only-to-onslide},\linebreak where I also first posted the solution that gave rise to this package.

%The problem of jumping figures 

A second functionality of the package is that overlay information of the \verb|tikz| macros in document classes other than \verb|beamer| are ignored. This allows the same code of a \verb|tikz| picture to be used in the paper and a presentation, or in an assignment and the lecture slides.

\section{Usage}

%The package does not provide any user-level macros. 
The functionality is added by loading the package with \verb|\usepckage{fikz}| in the preamble. If the current document is of the \verb|beamer| document class, loading the package will change the default overlay behavior of the commands \verb|\draw|, \verb|\fill|, \verb|\filldraw|, \verb|\clip|, and \verb|\node| from the \verb|\only| functionality to the \verb|\onslide| functionality. If the current document is of any other document class, overlay instructions \verb|<>| are ignored for all the above commands.

If the package is included in the definition of a document class that builds on \verb|beamer|, the package needs to be loaded with \verb|\RequirePackage[beamer]{fikz}|. If the package is included in the definition of any other document class, it needs to be loaded with \verb|\RequirePackage[doc]{fikz}|.

\section{Implementation}

The \verb|tikz| package redefines the commands \verb|\draw|, \verb|\fill|, \verb|\filldraw|, \verb|\clip|, and \verb|\node| at the beginning of every \verb|tikzpicture| environment so that their functionality is preserved even if the user defines their own macro with the same name. This package taps into this mechanism by parsing the overlay specification separately in this redefinition and then calling \verb|\onslide<>{<original definition>}| in the beamer document class and ignoring the overlay specification in other document classes.

The active document class is determined with \verb|\@ifclassloaded|. Since this macro is not available in the definition of a document class, the package needs to be loaded with options \verb|beamer| or \verb|doc| in this case, depending on which of the two functionalities is desired.

\end{document}
